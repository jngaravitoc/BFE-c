\documentclass[12pt]{article}

\usepackage[margin=1in]{geometry} 
\usepackage{amsmath,amsthm,amssymb}
\usepackage{graphicx}
\usepackage{caption}
\usepackage{float}

\begin{document}

\title{SCF Method applied to different halos}%replace X with the appropriate number
\author{Nicolas Garavito-Camargo} 

\maketitle

The smoothening $b_j$ have a maximum value of $b_j=0$ when the
variance is $0$ and the minimum value is $b_j=1$ when the coefficient
is very small.

When $b=0.5$ the coefficient the $S/N=1.0$.

When does the variance of a coefficient is smaller than the value
of the coefficient? in this case  $b$ can be larger than $0.5$

Following Weinberg (1996) the variance of the coefficients is defined
as:

\begin{equation}
var(\hat{a}_j) = \dfrac{1}{N}{E[\Psi_j\Psi_j] - E[\Psi_j]E[\Psi_j]}
\end{equation}

Which in Lowing (2011) notation would be:

\begin{equation}\label{eq:var}
var(S_{n,l,m}) = M \left[ \sum_{i}^{N}m_i \Psi_{n,l,m}^2(x_i)
 - \left(\sum_{i}^{N}m_i \Psi_{n,l,m}(x_i)\right)^2
\right]
\end{equation}

The first term in the right hand side is the covariance matrix and
the second term of Eq.\ref{eq:var} are the coefficients
$\hat{S}_{n,l,m}^2$ and $\hat{T}_{n,l,m}^2$. $M$ is the total
mass and $\Psi_{nlm}$ is defined as:

\begin{equation}
\Psi_{nlm}(x_i) = (2-\delta_{m,0})
\tilde{A}_{n,l}\Phi_{n,l}(r_i)Y_{l,m}(\theta_i)cos(m\phi_i)
\end{equation}

In order to do the PCA analysis we apply the method explained in
Weinber (1996). The first step is to compute 
the outer-product matrix defined as:

\begin{equation}
S_{kl} = \hat{a}_k \hat{a}_l
\end{equation}

Which in Lowing (2011) notation the matrix will take the form:

\begin{equation}
\tilde{S}_{nlm n'l'm'} = \hat{S}_{nlm} \hat{S}_{n'l'm'}
\end{equation}

The size of this matrix is going to be $(n\times l \times m)\times( n' \times l'
\times m')$.

The eigenvectors of $\tilde{S}_{nlm n'l'm'}$ are going to define the
rotation matrix $T_{(n' \times l' \times m')(n \times l \times m)}$.

The coefficients in the principal basis are going to be:

\begin{equation}
\hat{S}^{*}_{nlm} = \sum_{nlm} T_{n'l'm',nlm} S_{nlm}
\end{equation}

The variance of the coefficients in the principal basis would be:

\begin{equation}
var(\hat{S}^{*}_{nlm}) = \sum_{nlm,nlm} T_{ijk,nlm}
var(\hat{S}_{(nlm)(n'l'm')}) T_{(n'l'm',i'j'k')}
\end{equation}

Where $var(\hat{S}_{(nlm)(n'l'm')})$ is the covariance matrix.

Finally the smoothening of the coefficients $b^*_{nlm}$ in the
principal basis can be computed using:

\begin{equation}
b^*_{nlm} = \dfrac{1}{1 + var(\hat{S}^*_{nlm})/{S^*_{nlm}}^2}
\end{equation}

\section{NFW Halo}

\begin{figure}[H]
\centering
\includegraphics[scale=0.5]{../code/SCF/NFW_smooth.png}
\caption{Smooth coefficients for a NFW profile.}
\end{figure}

\section{Triaxial Halos}

\begin{figure}[H]
\centering
\minipage{0.33\textwidth}
\includegraphics[scale=0.4]{../code/SCF/spherical.png}
\endminipage
\minipage{0.33\textwidth}
\includegraphics[scale=0.4]{../code/SCF/Oblate.png}
\endminipage
\minipage{0.33\textwidth}
\includegraphics[scale=0.4]{../code/SCF/Prolate.png}
\endminipage
\caption{Potential contours of triaxial halos. The potential is
computed with Gadget.\label{fig:coeffMW}}
\end{figure}


\begin{figure}[H]
\centering
\minipage{0.33\textwidth}
\includegraphics[scale=0.4]{../code/SCF/sphresiduals.png}
\endminipage
\minipage{0.33\textwidth}
\includegraphics[scale=0.4]{../code/SCF/oblateresiduals.png}
\endminipage
\minipage{0.33\textwidth}
\includegraphics[scale=0.4]{../code/SCF/prolateresiduals.png}
\endminipage
\caption{Residuals contours of the potential of triaxial halos
computed with Gadget and the SCF method.
\label{fig:coeffMW}}
\end{figure}

\begin{figure}[H]
\centering
\minipage{0.25\textwidth}
\includegraphics[scale=0.3]{../code/SCF/Hernquistmono.png}
\endminipage
\minipage{0.25\textwidth}
\includegraphics[scale=0.3]{../code/SCF/Hernquistdip.png}
\endminipage
\minipage{0.25\textwidth}
\includegraphics[scale=0.3]{../code/SCF/Hernquistquad.png}
\endminipage
\minipage{0.25\textwidth}
\includegraphics[scale=0.3]{../code/SCF/Hernquistoct.png}
\endminipage
\caption{Potential multipole contours of a spherical halo
computed with the SCF method.\label{fig:coeffMW}}
\end{figure}


\begin{figure}[H]
\centering
\minipage{0.25\textwidth}
\includegraphics[scale=0.3]{../code/SCF/oblatemono.png}
\endminipage
\minipage{0.25\textwidth}
\includegraphics[scale=0.3]{../code/SCF/oblatedip.png}
\endminipage
\minipage{0.25\textwidth}
\includegraphics[scale=0.3]{../code/SCF/oblatequad.png}
\endminipage
\minipage{0.25\textwidth}
\includegraphics[scale=0.3]{../code/SCF/oblateoct.png}
\endminipage
\caption{Potential multipoles contours of an oblate halo
computed with the SCF method.\label{fig:coeffMW}}
\end{figure}



\begin{figure}[H]
\centering
\minipage{0.33\textwidth}
\includegraphics[scale=0.4]{../code/SCF/prolatemono.png}
\endminipage
\minipage{0.33\textwidth}
\includegraphics[scale=0.4]{../code/SCF/prolatedip.png}
\endminipage
\minipage{0.33\textwidth}
\includegraphics[scale=0.4]{../code/SCF/prolatequad.png}
\endminipage
\caption{Potential multipoles contours of a prolate halo
computed with the SCF method.\label{fig:coeffMW}}
\end{figure}

\begin{figure}[H]
\centering
\includegraphics[scale=0.5]{../code/SCF/spherical_smooth_coef}
\caption{Smoothen coefficients of the spherical halo}
\end{figure}


\begin{figure}[H]
\centering
\includegraphics[scale=0.5]{../code/SCF/oblate_smooth_coef}
\caption{Smoothen coefficients of the oblate halo}
\end{figure}


\begin{figure}[H]
\centering
\includegraphics[scale=0.5]{../code/SCF/prolate_smooth_coef}
\caption{Smoothen coefficients of the prolate halo}
\end{figure}

\begin{figure}[H]
\centering
\includegraphics[scale=0.5]{../code/SCF/spherical_PCA_coef}
\caption{Smoothen coefficients in the principal basis
for the spherical halo}
\end{figure}



\section{MW DM halo response to the LMC}


\begin{figure}[H]
\centering
\minipage{0.25\textwidth}
\includegraphics[scale=0.3]{../code/SCF/MWgadget.png}
\endminipage
\minipage{0.25\textwidth}
\includegraphics[scale=0.3]{../code/SCF/MWSCF.png}
\endminipage
\minipage{0.25\textwidth}
\includegraphics[scale=0.3]{../code/SCF/MWres.png}
\endminipage
\minipage{0.25\textwidth}
\includegraphics[scale=0.3]{../code/SCF/MWresout.png}
\endminipage
\caption{Potential contours of the MW dark matter
halo interacting with the LMC. The LMC halo is not showed
in the figure. Left panel shows the potential computed with
gadget. Middle panel shows the potential computed with the 
SCF method. Right panel shows the potential residuals between
the Gadget and the SCF computation.\label{fig:coeffMW}}
\end{figure}


\begin{figure}[H]
\centering
\minipage{0.33\textwidth}
\includegraphics[scale=0.4]{../code/SCF/MWmonopole.png}
\endminipage
\minipage{0.33\textwidth}
\includegraphics[scale=0.4]{../code/SCF/MWdipole.png}
\endminipage
\minipage{0.33\textwidth}
\includegraphics[scale=0.4]{../code/SCF/MWquad.png}
\endminipage
\caption{Potential multipoles of the MW dark matter
halo interacting with the LMC. Figure shows the monopole,
dipoele and quadrupolar contribution to the potential.}
\end{figure}



\end{document}
