\documentclass[14pt]{article}
\usepackage{amsmath}
\usepackage{listings} % For writing code see http://ctan.org/pkg/listings
\usepackage{graphicx}
\usepackage{float}
\usepackage[margin=1.0in]{geometry}
\usepackage{hyperref}
\usepackage{fancyhdr}

\pagestyle{fancy}

\title{BFE high values on the coefficients with high order $l$}
\author{Nico Garavito-Camargo}


\begin{document}
\maketitle

\section{Computing the coefficients review:}


\begin{table}[h]
  \centering
  \begin{tabular}{c  c}
    \hline
    \hline
    Hernquist+92 notation & Lowing+11 notation \\
    \hline
    $A_{nlm}$ & $S_{nlm} cos m\Phi + T_{nlm}sin m\Phi $\\
    $I_{nl} = 1/\tilde{A}_{nl}$ & $1/\tilde{A}_{nl}$\\
    $K_{nl}$ & $K_{nl}$ \\
    $\tilde{\rho}_{nl}$ & $\rho_{nl}$\\
    $\tilde{\Phi}_{nl}$ & $\Phi_{nl}$\\
    \hline
    \hline
  \end{tabular}
  \caption{Notation conversion between Hernquist92 and Lowing11 papers}
\end{table}


\begin{equation}
  \Phi_{nl}(r) = - \dfrac{r^l}{r(1+r)^{2l+3}}C_{n}^{2l+3/2}(\xi)\sqrt{4\pi}
\end{equation}



\begin{equation}
  \rho_{nl}(r) = \dfrac{K_{nl}}{2\pi}\dfrac{r^l}{(1+r)^{2l+1}}C_{n}^{2l+3/2}(\xi)\sqrt{4\pi}
\end{equation}

Where $K_{nl}$ is defined as:
\begin{equation}
    K_{nl}=\dfrac{1}{2}n(n+4l+3) +(l+1)(2l+1)
\end{equation}

\begin{equation}
  \rho(r, \theta, \phi) = \sum_{n} \sum_l \sum_m Y_{lm}(\theta) \rho_{nl}(r)
  \left[ S_{nlm} cos m\phi + T_{nlm} sin m \phi \right]
\end{equation}

\section{Test-example with a Hernquist halo:}







\end{document}

